\chapter{Conclusion}

Next steps are identified for work on the guidance law, and a development timeline is presented.



Both pathways are considered, and are to be worked on in parallel.

\textbf{Convergence Studies}

Determining limits on convergence of a feedback guidance law is challenging, but there may be value in sweeping large swaths of the parameter space to determine if guidance law works better for certain cases. Of key interest are orbits which are very closer to each other, such as for the GEO disposal case.

Figuring the issues with ``last-mile'' convergence would mark a significant improvement. A helpful approach in doing so may be to study the evolution of $Q$ in time, alongside plots of the orbital elements. As mentioned in Section \ref{sec:discussion}, a quantitative analysis of how often the sail operates in degraded or feathered mode could provide opportunities for improving the approach taken to regularize the steering angles.

\textbf{Steering Angle Regularization Heuristic Rework}

The current approach for steering angle regularization is simple and functional, but lacks a solid mathematical justification in its design. Designing a regularization scheme which is based on some form of mathematical optimality is desirable from a point of analysis, and may lead to better performance.

Coverstone \cite{coverstone2003technique} presents a feedback guidance law which maximizes the amount of thrust generated by a solar sail towards a given direction. Oguri \cite{oguri2023solar} makes use of a similar approach. The idea is to find some $\hat{n} = \hat{n}^\dagger$ such that the product $\vec{F}(\hat{n}) \cdot \hat{n}^*$ (i.e. the projection of the resultant acceleration upon the ideal steering direction) is maximized. This can yield an orientation which is neither exactly on the ``projection cone'' nor directly facing towards $\hat{n}^*$, but which actually maximizes the ``amount of progress'' made towards the ideal direction.

\textbf{Optimization}

Further optimization runs can be performed using the current implementation, and improved once the above changes are made.

Exploring a large variety of transfer cases may allow the discovery of trends in optimal weights (e.g. emphasizing $W_p$ when orbit raising is needed), such that a general heuristic can be created for tuning guidance weights in the absence of a simulator.

\subsection{Applications to New Dynamics}
The current dynamics model is very simple. Demonstration of robustness to plant model variation would be a very attractive attribute for the guidance law.

\textbf{Other Bodies}

Only Earth-centric orbits have been considered so far. With increasing interest in cislunar space, it may be useful to investigate the applicablity of solar sails for translunar trajectories.

There are also mission using solar sails to the inner solar system, such as a Mercury sample return mission proposed by Hughes \cite{hughes2006sample}. Although the guidance law developed in this project is targeted at planetocentric orbits, it may also work for heliocentric trajectories.

\textbf{Sail Thrust Models}

So far, only a flat idealized sail has been considered. Ref. \cite{oguri2023solar} uses a more sophisticated thrust model incorporated into the \textit{Q-Law}. It would be interesting to see how the guidance law produced in this project compares against Oguri's.

\section{Far Future Ideas}

These ideas are unsuitable for investigation in the scope of a thesis, but would be interesting for further work beyond the scope of this project.

\textbf{Non-Keplerian Orbits}

Solar sails are a prime candidate for exploiting non-Keplerian trajectories such as halo orbits around Lagrange points; the lack of a propellant consumption leads to an uncapped mission life in a frozen orbit. Applying the guidance law to a multibody problem could give interesting results, but is made difficult by the lack of orbital elements to describe such trajectories.

\textbf{As a Starting Point for Global Methods}

The trajectories produced by the guidance law could be used as an initial guess for global methods to kickstart their optimization processes.
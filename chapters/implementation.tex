\chapter{Implementation and Testing}

\lettrine{D}{etails} about the implementation of QUAIL

\section{Implementation in Simulation}
With reference to the architecture in Figure \ref{fig:algorithm_diagram} and the procedures shown above, the complete guidance law and dynamics are implemented in two separate simulators.

Simulation is used to obtain trajectories through the solution of the planetocentric solar sail feedback guidance problem. Several cases of the guidance problem are solved, and the resultant trajectories are then analyzed.

\subsection{MATLAB Implementation}
\href{https://github.com/itchono/SLyGA}{Link to GitHub Repository}

The MATLAB implementation, named ``SLyGA'' (Solar Lyapunov Guidance Algorithm) is the more fully featured simulator of the pair. SLyGA includes diagnostic metrics and plotting utilties to generate visualizations of trajectories.

MATLAB's built-in ODE integrators are mature and well-tested, and therefore SLyGA is used for high-fidelity simulation cases.

A key drawback of this implementation is its speed -- simulation runs take anywhere from tens of seconds to several minutes to complete. This makes it unsuitable for optimization of the guidance law.

\subsection{C/Python Implementation}
\href{https://github.com/itchono/cshanty}{Link to GitHub Repository}

The C/Python implementation, named ``cshanty'' (i.e. Sea Shanty) is built for tuning the weights of the guidance law. The C portion of the codebase contains all of the guidance law and dynamics, as well as a custom ODE integrator based on weights found by Verner \cite{verner2010numerically}. The Python portion of the code wraps the C simulator into an interface for optimization using SciPy \cite{2020SciPy-NMeth}.

\section{Trajectory Cases}
Table \ref{tab:trajectory_cases} gives four trajectory cases used to assess the performance of the guidance law.

\begin{table}[H]
  \centering
  \begin{tabular}{L{2.7cm}L{1.2cm}C{3cm}C{1cm}C{1cm}C{1cm}C{1cm}C{2cm}}
    \toprule
    \textbf{Case} &         & $p$ [\qty{1e3}{m}] & $f$       & $g$        & $h$       & $k$       & $L$ [\unit{rad}] \\
    \midrule
    GEO Disposal  & Initial & \num{42164}        & 0         & 0          & 0         & 0         & 0                \\
                  & Final   & \num{42464}        & 0         & 0          & Free      & Free      & N/A              \\
    Plane Change  & Initial & \num{20000}        & 0.5       & 0          & 1         & 0         & 0                \\                    & Final & \num{20000} & 0.5 & 0 & \num{-1} & 0 & N/A\\
    ``Benchmark'' & Initial & \num{20000}        & \num{0.5} & \num{-0.2} & \num{0.5} & \num{0}   & \num{0}          \\
                  & Final   & \num{25000}        & \num{0.2} & \num{0.5}  & \num{0}   & \num{0.3} & N/A              \\
    Polar GSO     & Initial & \num{11625}        & 0.725     & 0          & 0         & 0         & 0                \\
                  & Final   & \num{42165}        & 0         & 0          & 0         & \num{-1}  & N/A              \\
    \bottomrule
  \end{tabular}
  \caption{Trajectory cases.}
  \label{tab:trajectory_cases}
\end{table}
Free orbital elements are implemented by assigning a value of zero to the corresponding weight (i.e. $W_{\moe} = 0$).

For each trajectory case, values of the guidance parameters used are presented in Table \ref{tab:trajectory_case_params}.
\begin{table}[H]
  \centering
  \begin{tabular}{L{1.7cm}ccccccccc}
    \toprule
    \textbf{Case} & $W_P$ & $\gamma$ & $r_{p, \text{min}}$ (km) & $W_p$ & $W_f$ & $W_g$ & $W_h$ & $W_k$ & $\kappa$ \\
    \midrule
    GEO Disposal  & 0     & N/A      & N/A                      & 5     & 0.5   & 1     & 0     & 0     & \ang{70} \\
    Plane Change  & 1     & 1        & 10000                    & 1     & 1     & 1     & 1     & 1     & \ang{70} \\
    ``Benchmark'' & 0     & N/A      & N/A                      & 1     & 1     & 1     & 1     & 1     & \ang{70} \\
    Polar GSO     & 1     & 5        & 6878                     & 1     & 1     & 1     & 1     & 1     & \ang{70} \\
    \bottomrule
  \end{tabular}
  \caption{Guidance parameters for each case.}
  \label{tab:trajectory_case_params}
\end{table}

\section{Tuning of Guidance Algorithm}
The parameters in Table \ref{tab:trajectory_case_params} are a baseline obtained from manual tuning. Optimization is performed to find the parameters yielding minimum time-of-flight.

\subsection{Thrust Angle Heuristic Tuning}
The value of $\kappa$ is optimized using bounded univariate optimization through measured time-of-flight in the  Benchmark case. The value of $\kappa$ in the interval $[\ang{30}, \ang{85}]$ is determined using Brent's method \cite{brent2013algorithms}.

The resultant optimum of $\kappa \approx \ang{64}$ is used for proceeding analysis of all the other cases. Note that this is not an exact optimum for all of the other cases, and further tuning is to be done on a case-by-case basis.

\subsection{Orbital Element Weight Tuning}
All five weights $W_{\moe}, \ \moe \in \{p, f, g, h, k\}$ are optimized using a simulated-annealing global optimization algorithm. The computationally efficient implementation of the simulation in C allows for the global optimization runs to perform thousands of iterations in under one hour.

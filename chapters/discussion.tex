\chapter{Discussion}
\label{chp:discussion}

\lettrine{T}{he} performance of the guidance law is assessed based on performance in numerical experiments. Consideration of the methodology is used to explain some of the observed behaviours, and deficiencies are noted for improvement.

The improvement in time-of-flight (and number of revolutions) is remarkable. By inspecting the plots of the orbital elements, it is clear that adjusting the weights of the guidance changes the ``strategy'' employed by the guidance law.

By emphasizing certain elements over others, the guidance law is more willing to accept an increase in error in one element for a reduction in another. This is exemplified in Figure \ref{fig:results_benchmark_optim_a}.

By inspecting the trajectory plots, it is evident that the optimized transfers ``waste fewer actions'' compared to their baseline counterparts. In the Benchmark case, the total \(\Delta v\) expenditures are very similar, but the trajectory taken by the optimized tuning looks far more direct than the baseline case.
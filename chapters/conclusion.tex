\chapter{Significance and Future Work}

\lettrine{A}{n} overview of the developed guidance law is presented, and potential areas for future work are identified.

\section{Significance of QUAIL}
QUAIL is a useful guidance law both for practical purposes, and also as an academically appealing demonstration of producing a guidance law for solar sails.

\subsection{As a Product}
A solar sailing guidance law for planetocentric maneuvering was developed, proving to be successful in a limited run of orbital transfer cases. This achievement in itself is significant, as there are very few examples of other guidance laws which can be used for this purpose.

The novel approach taken shows great promise for applicability to other areas. As shown in the extended studies, QUAIL can readily be modified and applied to different problems, making it excellent as both a mission planning tool and a design tool for developing new solar sail configurations. Since QUAIL does not explicitly depend on the thrust models of the solar sail, it can facilitate trajectory analysis for unique and esoteric spacecraft designs, and will remain applicable as better reflection models for solar sails are developed.

The two main barriers holding back QUAIL from being dropped into a real spacecraft today are the lack of testing and lack of consideration for solar sail attitude dynamics. However, further extension and development is highly likely to produce a workable solution.

\subsection{As an Approach}
The idea of ignoring solar sail dynamics in the Q-Law is highly unusual, but produced amazingly good results with a very simple formulation. QUAIL is definitely not elegant by any measure, but it is simple and effective. It is quite surprising that the approach of cascading a simple low-thrust guidance law with an additional stage for cone angle limits has not been attempted more widely.

This is very good news for the solar sail guidance community; being able to adapt the rich pool of research in conventional low-thrust guidance for solar sails means that significant progress could be made in the field by applying the approach taken for QUAIL to other low-thrust guidance laws. Bridging the gap between solar-sail-specific guidance research and the broader field of low-thrust guidance research can help the two communities work together to develop even better guidance laws in the future.

\section{Future Work}
An overview of the key directions for future work is discussed.
\subsection{Outstanding Questions}
The path forward for QUAIL can be summarized by the following questions:
\begin{itemize}
    \item Can be formulation of QUAIL be adjusted to get better performance while still being relatively simple?
    \item How well will QUAIL work in simulations with a state of the art solar sail thrust model?
    \item Can it be used around other planets/bodies? e.g. Earth-Moon transfers
    \item Can it target things other than Keplerian orbits? e.g. Lagrange points, pole-sitting missions
    \item Can the idea of cone angle limiting be applied to other solar sail problems, or even conventional low-thrust spacecraft?
    \item Can QUAIL solutions be used as a first guess for global optimization methods?
\end{itemize}
The main areas for improvement are then 1) Formulation 2) Dynamics and 3) Applications to New Problems.

\subsection{Formulation Improvements}
The formulation of QUAIL is simple and quite generic. Without adding too much additional complexity, some changes could be made to make it more performant.

\textbf{Cone Angle Adaptation Heuristic Rework}

The current approach for cone angle adaptation is simple and functional, but lacks a solid mathematical justification in its design. Designing an adaptation scheme which is based on some form of mathematical optimality is desirable from a point of analysis, and may lead to better performance.

Coverstone \cite{coverstone2003technique} presents a feedback guidance law which maximizes the amount of thrust generated by a solar sail towards a given direction. Oguri \cite{oguri2023solar} makes use of a similar approach. The idea is to find some \(\hat{n} = \hat{n}^\dagger\) such that the product \(\vec{F}(\hat{n}) \cdot \hat{n}^*\) (i.e. the projection of the resultant acceleration upon the ideal steering direction) is maximized. This can yield an orientation which is neither exactly on the ``projection cone'' nor directly facing towards \(\hat{n}^*\), but which actually maximizes the ``amount of progress'' made towards the ideal direction.

\textbf{Higher Fidelity Q-Law Derivatives}

The maximum rates of change for \(f\) and \(g\) are approximated in the Q-Law according to the formulation by Ref. \cite{vargaperez2016}. However, Ref. \cite{sanjeev2023} shows that a semi-analytical formulation can be readily obtained which gives an improved estimate.

\subsection{Higher Fidelity Dynamics}
The current dynamics model is very simple. Demonstration of robustness to plant model variation would be a very attractive attribute for the guidance law.

\textbf{Solar Sail Thrust Models}

So far, only a flat idealized sail has been used in simulations. Ref. \cite{oguri2023solar} uses a more sophisticated thrust model incorporated into the Q-Law. It would be interesting to see how the guidance law produced in this project compares against Oguri's. A key advantage of QUAIL is that it can be applied to any solar sail thrust model without altering its formulation.

Testing QUAIL with a realistic thrust model may reveal issues in either the Q-Law or cone angle adaptation stage related to the fact that thrust is not produced directly towards the sail normal.

\textbf{Incorporation of Attitude Dynamics}

Attitude dynamics have been ignored for this project, but are essential for a guidance law to be usable. Adding an extra layer onto the simulator which accounts for attitude dynamics would be a simple way of testing QUAIL as-is.

A more interesting development would be to reformulate the second stage of the guidance law to consider not only the direction of the Sun, but also the agility of the solar sail. For example, there could be a second cone introduced, of width \(\omega_{\text{max}} \Delta t\) pointing in the direction \(\hat{n}_{k-1}\), representing the attitude space that the solar sail can traverse in one timestep.


\subsection{Applications to New Problems}
This thesis is concerned only with geocentric orbits, but there is nothing stopping QUAIL from being applied to trajectories around other planets or even moons of planets.

\textbf{Other Bodies}

With increasing interest in cislunar space, it may be useful to investigate the applicability of solar sails for translunar trajectories.

There are also mission using solar sails to the inner solar system, such as a Mercury sample return mission proposed by Hughes \cite{hughes2006sample}. Although the guidance law developed in this project is targeted at planetocentric orbits, it may also work for heliocentric trajectories.

\textbf{Non-Keplerian Orbits}

Solar sails are a prime candidate for exploiting non-Keplerian trajectories such as halo orbits around Lagrange points; the lack of a propellant consumption leads to an uncapped mission life in a frozen orbit. Applying the guidance law to a multibody problem could give interesting results, but is made difficult by the lack of orbital elements to describe such trajectories.

\textbf{As a Starting Point for Global Methods}

Many Q-Law papers describe the possibility of using solutions obtained by their guidance law to initialize a global method, \textit{but no one appears to have actually done so}. The trajectories produced by QUAIL could be used as an initial guess for global methods to kickstart their optimization processes. This would address two important research gaps:

\begin{enumerate}
    \item Using Lyapunov methods (i.e. Q-Laws) to generate initial guesses for high-fidelity trajectory optimization is a frequently mentioned but unexplored direction.
    \item Generating good initial guesses for global optimization of solar sail trajectories is difficult.
\end{enumerate}

A Q-Law (either QUAIL, its predecessors, or a derivative) could be used in conjunction with a simulator to integrate an initial trajectory forwards in time to reach a final target orbit. The position history flown by the Q-Law can then be used as the starting point for a pseudospectral global trajectory optimization scheme. Q-Law solutions are cheap to evaluate because the cost is about as much as propagating an orbit in time. Using such a solution to initialize a global optimization run will save expensive iterations, and could result in better convergence characteristics.

\section{Life Lessons from a Guidance Law}
To close off this thesis, the 5 most important points discussed in solving the planetocentric guidance problem for solar sails are listed below, along with an accompanying life lesson.

\begin{enumerate}
    \item Use feedback guidance instead of solving for a global trajectory. \newline \textit{Take things one step at a time.}
    \item Assume a solar-sail-agnostic thrust model for the Q-Law. \newline \textit{Keep it simple.}
    \item Adapt the solution from the Q-Law to make it work for solar sails.\newline \textit{Be willing to try something unconventional.}
    \item Relax Lyapunov stability. \newline \textit{Sometimes it must get worse before it gets better.}
    \item Adjust the strategy of the guidance law to set up future maneuvers in a favourable position. \newline \textit{Set up others for success.}
\end{enumerate}
\chapter{Significance and Future Work}

\lettrine{A}{n} overview of the developed guidance law is presented, and potential areas for future work are identified.


\section{Importance of Guidance Law}

\subsection{As a Product}
A solar sailing guidance law for planetocentric maneuvering was developed, proving to be successful in a limited run of orbital transfer cases. Several issues remain, but the novel approach taken shows great promise for applicability to other areas, and the guidance law is easy to modify for further improvements.

\subsection{As an Approach}
Cone angle adaptation



\section{Future Work}

\subsection{Considerations for Higher Fidelity Dynamics}


The current dynamics model is very simple. Demonstration of robustness to plant model variation would be a very attractive attribute for the guidance law.

\textbf{Sail Thrust Models}

So far, only a flat idealized sail has been considered. Ref. \cite{oguri2023solar} uses a more sophisticated thrust model incorporated into the Q-Law. It would be interesting to see how the guidance law produced in this project compares against Oguri's.


\textbf{Incorporation of Attitude Dynamics}

Attitude dynamics has been ignored for this project.

\textbf{Cone Angle Adaptation Heuristic Rework}

The current approach for cone angle adaptation is simple and functional, but lacks a solid mathematical justification in its design. Designing an adaptation scheme which is based on some form of mathematical optimality is desirable from a point of analysis, and may lead to better performance.

Coverstone \cite{coverstone2003technique} presents a feedback guidance law which maximizes the amount of thrust generated by a solar sail towards a given direction. Oguri \cite{oguri2023solar} makes use of a similar approach. The idea is to find some \(\hat{n} = \hat{n}^\dagger\) such that the product \(\vec{F}(\hat{n}) \cdot \hat{n}^*\) (i.e. the projection of the resultant acceleration upon the ideal steering direction) is maximized. This can yield an orientation which is neither exactly on the ``projection cone'' nor directly facing towards \(\hat{n}^*\), but which actually maximizes the ``amount of progress'' made towards the ideal direction.



\subsection{Applications to New Problems}


\textbf{Other Bodies}

Only Earth-centric orbits have been considered so far. With increasing interest in cislunar space, it may be useful to investigate the applicablity of solar sails for translunar trajectories.

There are also mission using solar sails to the inner solar system, such as a Mercury sample return mission proposed by Hughes \cite{hughes2006sample}. Although the guidance law developed in this project is targeted at planetocentric orbits, it may also work for heliocentric trajectories.

\textbf{Non-Keplerian Orbits}

Solar sails are a prime candidate for exploiting non-Keplerian trajectories such as halo orbits around Lagrange points; the lack of a propellant consumption leads to an uncapped mission life in a frozen orbit. Applying the guidance law to a multibody problem could give interesting results, but is made difficult by the lack of orbital elements to describe such trajectories.

\textbf{As a Starting Point for Global Methods}

Many Q-Law papers describe the possibility of using solutions obtained by their guidance law to initialize a global method, \textit{but no one appears to have actually done so}. The trajectories produced by QUAIL could be used as an initial guess for global methods to kickstart their optimization processes. This would address two important research gaps:

\begin{enumerate}
    \item Using Lyapunov methods (i.e. Q-Laws) to generate initial guesses for high-fidelity trajectory optimization is a frequently mentioned but unexplored direction.
    \item Generating good initial guesses for global optimization of \textit{solar sail} trajectories is difficult.
\end{enumerate}

A Q-Law (either QUAIL, its predecessors, or a derivative) could be used in conjunction with a simulator to integrate an initial trajectory forwards in time to reach a final target orbit. The position history flown by the Q-Law can then be used as the starting point for a pseudospectral global trajectory optimization scheme. Q-Law solutions are cheap to evaluate because the cost is about as much as propagating an orbit in time. Using such a solution to initialize a global optimization run will save expensive iterations, and could result in better convergence characteristics.

\section{Summary of Outstanding Questions}
\begin{itemize}
    \item How well will QUAIL work in simulations with a state of the art solar sail thrust model?
    \item Can it be used around other planets/bodies? e.g. Earth-Moon transfers
    \item Can it target things other than Keplerian orbits? e.g. Lagrange points, pole-sitting missions
    \item Can the idea of cone angle limiting be applied to other solar sail problems, or even conventional low-thrust spacecraft?
    \item Can QUAIL solutions be used as a first guess for global optimization methods?
\end{itemize}

\section{Life Lessons from a Guidance Law}
To send off this thesis, the 4 major keypoints of QUAIL are presented, along with an accompanying life lesson.

\begin{enumerate}
    \item Use feedback guidance instead of solving for a global trajectory. \newline \textit{Take things one step at a time.}
    \item Assume a solar-sail-agnostic thrust model for the Q-Law. \newline \textit{Keep it simple.}
    \item Adapt the solution from the Q-Law to make it work for solar sails.\newline \textit{Be willing to try something unconventional.}
    \item Relax Lyapunov stability for better overall performance. \newline \textit{Sometimes it must get worse before it gets better.}
\end{enumerate}